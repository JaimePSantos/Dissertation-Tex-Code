\documentclass[../../dissertation.tex]{subfiles}
\begin{document}

Quantum computing is an emergent field that brings together Quantum Mechanics,
Computer Science and Information Theory, which promises improvements to
classical algorithms such as simulation of quantum systems, cryptography, data
base searching and many others.  Among these algorithms, quantum walks may provide
a quadratic speed up when compared to their classical  counterparts,  allowing
improvements to applications such as element distinctness, searching problems,
matrix product verification and hitting times in graphs.  The present work
offers a general theoretical overview, simulation and circuit implementation of
the coined, staggered and continuous-time quantum walk models. The first two
chapters of this thesis are dedicated to the definition of the theoretical
framework, simulation in Python and comparison of the aforementioned quantum
walk models for the simple case of the dynamics in a line graph and for the
search algorithm in a complete graph. This is then used as a benchmark
for the final chapter, devoted to building and testing the circuits corresponding to models mentioned above in IBM's Qiskit. A main contribution of this dissertation concerns the circulant graph approach to diagonal operators for continuous-time quantum walks.
%TODO: Mencionar que fazemos comparacao entre os modelos, vantagens e
%desvantagens.
%TODO: Analise do ponto de vista teorico, do ponto de vista de simular que funciona como benchmark, analise de circuito de cada um e implementacao num computador NISQ de supercondutor.
%\keywords{master thesis}
\end{document}
