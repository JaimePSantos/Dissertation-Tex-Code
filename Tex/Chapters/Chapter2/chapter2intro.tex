\documentclass[../../dissertation.tex]{subfiles}
\begin{document}

The structure of this chapter is as follows. The first section presents a brief
introduction to random walks by reviewing the classical case. The following
sections are dedicated to the study of several quantum random walk models and
their advantages. This analysis is done by firstly describing the theoretical
framework for these walks, and then simulating them in Python, with code that
can be found on \textit{Github}
\footnote{https://github.com/JaimePSantos/QWSimulations}.\par 

Section \ref{sec:chap3Coinedwalk} introduces the quantum case of random walks
by analyzing the dynamics of the coined model, making use of Python's plot
capabilities in order to visualize the probability distributions associated
with this algorithm. The quantum walk is generally said to be
\textit{quadratically} faster than the classical one, which is reflected on the
behavior of the standard deviation associated with the probability
distribution, and in this section a brief comparison of this metric is also
presented.  Section \ref{sec:chap3StagWalk} is dedicated to the study of
another instance of a discrete quantum walk, but where a coin is not needed.
Instead, \textit{tessellations} are used to construct the Hamiltonians of this
algorithm, hence the name \textit{staggered}, effectively reducing the
associated Hilbert space. Here, plots are used to study the propagation of the
walk, and the effects of altering available the parameters will also be
analyzed.  Lastly, section \ref{sec:chap3Contwalk} presents the continuous-time
model of the quantum walk. Following the previous sections, the probability
distributions will be plotted for the different parameters that can be altered.
Finally, this chapter is closed with some final thoughts and remarks.
\end{document}
