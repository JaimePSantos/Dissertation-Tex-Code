\documentclass[../../dissertation.tex]{subfiles}
\begin{document}

The Church-Turing thesis, that states that any algorithmic process can
be simulated efficiently using a Turing Machine, was challenged by
\cite{solvaystrassen77} where they presented what is known as the
\textit{Solovay-Strassen primality test}. They showed that it is possible to
test whether a integer is prime or composite using a randomized algorithm. The
implication is that, because of the randomness, the Solvay-Strassen primality
test does not determine with certainty whether a integer is prime or composite,
rather it computes that a number is \textit{probably} prime or else
\textit{certainly} composite. This is of significance since no deterministic
test for primality was known at the time\footnote{Work by \cite{agrawal02} has since found a general, polynomial, deterministic, and unconditional primality proving algorithm.}, meaning that this was
an example of a class of problems that could not be efficiently solved by a
conventional deterministic Turing Machine.\par

%TODO: Talvez exemplos diferentes?  
%TODO: Quicksort faz sentido quando tamos a falar do monte carlo? Se calhar falar mais um bocado do algoritmo.
This led to a modification of the Church-Turing thesis, now stating that any
algorithm can be simulated efficiently using a \textit{probabilistic} Turing
machine. The discovery of more instances of such algorithms followed,
\cite{motwani1995} and \cite{papadimitrious1994} show several problems that can
be solved based on randomized algorithms. For example, the \textit{Quicksort}
algorithm, developed by \cite{hoare61}, has a high probability of finishing in
$O(n \log{n})$. In contrast to many deterministic algorithms that require
$O(n^2)$ time. They also show algorithms that take advantage of \textit{Markov
chains} and the \textit{Monte Carlo method}. The volume of a convex body,
proposed by \cite{dyer1991}, can be estimated by a randomized algorithm in
polynomial time; the permanent of a nonnegative entry matrix can also be
approximately calculated in probabilistic polynomial time as was shown by
\cite{jerrum2001} and the \textit{k-SAT} and satisfiability with restrictions
problem by \cite{schoning1999}. \par

%TODO: Definicao da wikipedia um bocado fraca
%TODO: Ref brownian motion : D ben avraham, s havlin , diffusion and reactions in fractals and disordered systems.
Random walks, as the name suggests, belong to this class of algorithms.
\cite{kpearson1905} coined the term random walk, and they can be described as
path consisting of a succession of steps determined by a stochastic process,
over a mathematical space. They are a special case of \textit{Markov chains},
which are stochastic processes that assume discrete values and whose next state
is dictated by a deterministic or random rule based only on the current state.
This is a useful framework, since it can be used to explain the behaviour of
systems across many fields, from the Brownian movement of particles moving
through a gas, to the price of a fluctuating stock as shown by
\cite{cootner67}.\par

The quantum analogue of the random walk was firstly developed by
\cite{aharonov1993}, where they defined the \textit{coined quantum random
walk}.  This model consists of a walker and a coin that determines the movement
of the walker, which are both quantum systems where time is a discrete variable
dictated by the successive quantum coin flips and shifts in position.
\cite{nayak2000} and \cite{aharonov2002}  present the first analyses of the
quantum walk on a graph described by a line. Further work by \cite{inui2003}
studies the behaviour of the walk on grids and \cite{aharonov2002} on general
regular graphs. The first algorithmic applications appear in the work of
\cite{shenvi2002} where they constructed a search problem based on the quantum
random walk, and \cite{ambainis2003} applied it to the element distinction
problem. On a more theoretical note, \cite{konno2002} demonstrated how the
classical and quantum models of the random walk on the line differ, and
\cite{grimmett2003} generalized this to higher dimensions. \cite{lovett2010} demonstrated
that any quantum algorithm can be reformulated as a discrete time quantum walk
algorithm, effectively showing that this model can be used for universal quantum
computation.\par

%TODO: Sera importante mencionar Dam, Mosca e Vazirani? >> Ler este artigo
%TODO: Melhorar o inicio.
%TODO: chakraborty? 
A different model for quantum random walks emerged from the work of
\cite{farhi1996}, where a different way of computing a search problem was
presented. They showed that evolving a system in time between an initial and
final Hamiltonian is analogous to the Grover algorithm, but continuous in time.
\cite{farhi2000} revised their work, now known as an adiabatic evolution, to
solve Boolean \textit{sat} problems. \cite{childs2004} formulated a model of a
quantum walk in terms of adiabatic evolution, known as \textit{continuous time
quantum walk} or \textit{adiabatic quantum walk}. \cite{aharonov2005} showed
that any quantum algorithm can be efficiently simulated using adiabatic
evolution, meaning that it is polynomially equivalent to the conventional
quantum computation model. Further work by \cite{childs2009} showed that this
model is indeed universal. The main difference between these two quantum walk
models lies on the fact that in the discrete case, the system evolves with
the flipping of a coin and subsequent movement of the walker, whilst in the
adiabatic case the system evolves smoothly in time.\par 

Yet another way of thinking about quantum walks was announced by
\cite{szegedy2004}, where he describes a discrete model based on Markov chain
random walks. At the foundation of this model is the duplication of a graph, a
process by which a bipartite graph is created. \cite{magniez2005} show how to
use this walk for triangle detection in an undirected graph. \cite{magniez2006}
proposed a search problem, which takes advantage of ergodicity and simmetry
properties of Markov chains, with quadratic gain compared to classical
algorithms. Problems like element distinctness, matrix product verification and
others were formulated within this framework by \cite{santha2008}. Further work
by \cite{portugal2015} established a connection between the coined and Szegedy's quantum
walk, by defining a model that encompasses and expands the latter.\par 

%TODO: Perceber a ligacao entre patel e staggered.
%TODO: Uma forma alternativa de construir uma quantum walk sem moeda de forma sistematica e atraves do staggered.
\cite{patel2004} pointed that, at the time, there was confusion surrounding the
scaling behaviour of discrete and adiabatic quantum walk algorithms. They
argued that this was because the former model used a coin, which is an extra
resource, and the latter didn't. So, in an attempt to resolve this confusion,
they showed that a discrete time quantum walk could be constructed without the
use of a coin. 
A new way of thinking about quantum walks came with the
development of the methods behind the construction of evolution operators, by
\cite{falk2013}, introducing the concept of \textit{tessellations}, based on
local diffusion operators. \cite{portugal2014} studied this model applied to a
line graph.
Further work by \cite{portugal2015b} formalized this approach naming it
\textit{staggered quantum walk}, and showed instances where the Szegedy quantum
walk is equivalent. This suggests that this is a more general model, being able
to describe other discrete-time quantum walks. \cite{portugal2017b} delved
deeper into this topic, adding Hamiltonians to the model, and
\cite{portugal2017a} shows how this can be instanced as a search problem in a
grid. Finally, \cite{coutinho2017} analyze how a continuous-time quantum walk
can be cast into this discrete model and \cite{moqadam2016} show a
possible physical implementation of this walk.


\end{document}
